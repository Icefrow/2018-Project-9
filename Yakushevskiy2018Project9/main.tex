\documentclass{llncs}
\usepackage[T2A]{fontenc}
\usepackage[utf8]{inputenc}
\usepackage[russian]{babel} 
\usepackage{graphicx}
\usepackage{natbib}

\title{Распознавание текста на основе скелетного представления толстых линий и сверточных сетей }

\author{Якушевский Н.О.}
\institute{Московский физико-технический институт (Государственный университет) \\ 
\email{yakushevskii.no@phystech.edu}}


\begin{document}

\maketitle

\begin{abstract}
	В данной работе исследуется задача распознования текста на изображении. Рассматриваются два подхода к решению этой задачи: на основе скелетного представления толстых линий и на основе свёрточных нейронных сетей. В качестве прикладной задачи предлагается классификация букв латинского алфавита на бинарном изображении. Точность классификации алгоритма на основе скелетного представления сравнивается с точностью алгоритма на основе свёрточных сетей.
\end{abstract}

\textit{Ключевые слова: распознавание текста, скелетное представление, сверточные нейронные сети}


\end{document}